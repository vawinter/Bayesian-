% Options for packages loaded elsewhere
\PassOptionsToPackage{unicode}{hyperref}
\PassOptionsToPackage{hyphens}{url}
%
\documentclass[
]{article}
\title{Homework3}
\author{V. Winter}
\date{3/14/2023}

\usepackage{amsmath,amssymb}
\usepackage{lmodern}
\usepackage{iftex}
\ifPDFTeX
  \usepackage[T1]{fontenc}
  \usepackage[utf8]{inputenc}
  \usepackage{textcomp} % provide euro and other symbols
\else % if luatex or xetex
  \usepackage{unicode-math}
  \defaultfontfeatures{Scale=MatchLowercase}
  \defaultfontfeatures[\rmfamily]{Ligatures=TeX,Scale=1}
\fi
% Use upquote if available, for straight quotes in verbatim environments
\IfFileExists{upquote.sty}{\usepackage{upquote}}{}
\IfFileExists{microtype.sty}{% use microtype if available
  \usepackage[]{microtype}
  \UseMicrotypeSet[protrusion]{basicmath} % disable protrusion for tt fonts
}{}
\makeatletter
\@ifundefined{KOMAClassName}{% if non-KOMA class
  \IfFileExists{parskip.sty}{%
    \usepackage{parskip}
  }{% else
    \setlength{\parindent}{0pt}
    \setlength{\parskip}{6pt plus 2pt minus 1pt}}
}{% if KOMA class
  \KOMAoptions{parskip=half}}
\makeatother
\usepackage{xcolor}
\IfFileExists{xurl.sty}{\usepackage{xurl}}{} % add URL line breaks if available
\IfFileExists{bookmark.sty}{\usepackage{bookmark}}{\usepackage{hyperref}}
\hypersetup{
  pdftitle={Homework3},
  pdfauthor={V. Winter},
  hidelinks,
  pdfcreator={LaTeX via pandoc}}
\urlstyle{same} % disable monospaced font for URLs
\usepackage[margin=1in]{geometry}
\usepackage{color}
\usepackage{fancyvrb}
\newcommand{\VerbBar}{|}
\newcommand{\VERB}{\Verb[commandchars=\\\{\}]}
\DefineVerbatimEnvironment{Highlighting}{Verbatim}{commandchars=\\\{\}}
% Add ',fontsize=\small' for more characters per line
\usepackage{framed}
\definecolor{shadecolor}{RGB}{248,248,248}
\newenvironment{Shaded}{\begin{snugshade}}{\end{snugshade}}
\newcommand{\AlertTok}[1]{\textcolor[rgb]{0.94,0.16,0.16}{#1}}
\newcommand{\AnnotationTok}[1]{\textcolor[rgb]{0.56,0.35,0.01}{\textbf{\textit{#1}}}}
\newcommand{\AttributeTok}[1]{\textcolor[rgb]{0.77,0.63,0.00}{#1}}
\newcommand{\BaseNTok}[1]{\textcolor[rgb]{0.00,0.00,0.81}{#1}}
\newcommand{\BuiltInTok}[1]{#1}
\newcommand{\CharTok}[1]{\textcolor[rgb]{0.31,0.60,0.02}{#1}}
\newcommand{\CommentTok}[1]{\textcolor[rgb]{0.56,0.35,0.01}{\textit{#1}}}
\newcommand{\CommentVarTok}[1]{\textcolor[rgb]{0.56,0.35,0.01}{\textbf{\textit{#1}}}}
\newcommand{\ConstantTok}[1]{\textcolor[rgb]{0.00,0.00,0.00}{#1}}
\newcommand{\ControlFlowTok}[1]{\textcolor[rgb]{0.13,0.29,0.53}{\textbf{#1}}}
\newcommand{\DataTypeTok}[1]{\textcolor[rgb]{0.13,0.29,0.53}{#1}}
\newcommand{\DecValTok}[1]{\textcolor[rgb]{0.00,0.00,0.81}{#1}}
\newcommand{\DocumentationTok}[1]{\textcolor[rgb]{0.56,0.35,0.01}{\textbf{\textit{#1}}}}
\newcommand{\ErrorTok}[1]{\textcolor[rgb]{0.64,0.00,0.00}{\textbf{#1}}}
\newcommand{\ExtensionTok}[1]{#1}
\newcommand{\FloatTok}[1]{\textcolor[rgb]{0.00,0.00,0.81}{#1}}
\newcommand{\FunctionTok}[1]{\textcolor[rgb]{0.00,0.00,0.00}{#1}}
\newcommand{\ImportTok}[1]{#1}
\newcommand{\InformationTok}[1]{\textcolor[rgb]{0.56,0.35,0.01}{\textbf{\textit{#1}}}}
\newcommand{\KeywordTok}[1]{\textcolor[rgb]{0.13,0.29,0.53}{\textbf{#1}}}
\newcommand{\NormalTok}[1]{#1}
\newcommand{\OperatorTok}[1]{\textcolor[rgb]{0.81,0.36,0.00}{\textbf{#1}}}
\newcommand{\OtherTok}[1]{\textcolor[rgb]{0.56,0.35,0.01}{#1}}
\newcommand{\PreprocessorTok}[1]{\textcolor[rgb]{0.56,0.35,0.01}{\textit{#1}}}
\newcommand{\RegionMarkerTok}[1]{#1}
\newcommand{\SpecialCharTok}[1]{\textcolor[rgb]{0.00,0.00,0.00}{#1}}
\newcommand{\SpecialStringTok}[1]{\textcolor[rgb]{0.31,0.60,0.02}{#1}}
\newcommand{\StringTok}[1]{\textcolor[rgb]{0.31,0.60,0.02}{#1}}
\newcommand{\VariableTok}[1]{\textcolor[rgb]{0.00,0.00,0.00}{#1}}
\newcommand{\VerbatimStringTok}[1]{\textcolor[rgb]{0.31,0.60,0.02}{#1}}
\newcommand{\WarningTok}[1]{\textcolor[rgb]{0.56,0.35,0.01}{\textbf{\textit{#1}}}}
\usepackage{graphicx}
\makeatletter
\def\maxwidth{\ifdim\Gin@nat@width>\linewidth\linewidth\else\Gin@nat@width\fi}
\def\maxheight{\ifdim\Gin@nat@height>\textheight\textheight\else\Gin@nat@height\fi}
\makeatother
% Scale images if necessary, so that they will not overflow the page
% margins by default, and it is still possible to overwrite the defaults
% using explicit options in \includegraphics[width, height, ...]{}
\setkeys{Gin}{width=\maxwidth,height=\maxheight,keepaspectratio}
% Set default figure placement to htbp
\makeatletter
\def\fps@figure{htbp}
\makeatother
\setlength{\emergencystretch}{3em} % prevent overfull lines
\providecommand{\tightlist}{%
  \setlength{\itemsep}{0pt}\setlength{\parskip}{0pt}}
\setcounter{secnumdepth}{-\maxdimen} % remove section numbering
\ifLuaTeX
  \usepackage{selnolig}  % disable illegal ligatures
\fi

\begin{document}
\maketitle

\hypertarget{question-1}{%
\subsection{Question 1}\label{question-1}}

Let 𝑦𝑖 be the number of sick days that a person takes due to an illness,
Let 𝑥𝑖 be the number of months that person has been taking part in a
treatment program.

Data model : yi \textasciitilde{} Pois(exp\{a + bxi\}) a Poisson
regression model with two regression parameters a and b.

We will assume that both of these regression parameters have prior
distributions that are Gaussian with mean zero and variance equal to 4.

\begin{Shaded}
\begin{Highlighting}[]
\NormalTok{x}\OtherTok{=}\FunctionTok{c}\NormalTok{(}\DecValTok{8}\NormalTok{,}\DecValTok{14}\NormalTok{,}\DecValTok{11}\NormalTok{,}\DecValTok{7}\NormalTok{,}\DecValTok{32}\NormalTok{,}\DecValTok{8}\NormalTok{,}\DecValTok{28}\NormalTok{,}\DecValTok{21}\NormalTok{,}\DecValTok{27}\NormalTok{,}\DecValTok{15}\NormalTok{,}\DecValTok{26}\NormalTok{,}\DecValTok{13}\NormalTok{,}\DecValTok{19}\NormalTok{,}\DecValTok{22}\NormalTok{,}\DecValTok{15}\NormalTok{,}
    \DecValTok{12}\NormalTok{,}\DecValTok{15}\NormalTok{,}\DecValTok{7}\NormalTok{,}\DecValTok{9}\NormalTok{,}\DecValTok{15}\NormalTok{,}\DecValTok{26}\NormalTok{,}\DecValTok{22}\NormalTok{,}\DecValTok{16}\NormalTok{,}\DecValTok{12}\NormalTok{,}\DecValTok{6}\NormalTok{)}

\NormalTok{y}\OtherTok{=}\FunctionTok{c}\NormalTok{(}\DecValTok{5}\NormalTok{,}\DecValTok{2}\NormalTok{,}\DecValTok{5}\NormalTok{,}\DecValTok{4}\NormalTok{,}\DecValTok{1}\NormalTok{,}\DecValTok{3}\NormalTok{,}\DecValTok{0}\NormalTok{,}\DecValTok{2}\NormalTok{,}\DecValTok{1}\NormalTok{,}\DecValTok{2}\NormalTok{,}\DecValTok{2}\NormalTok{,}\DecValTok{5}\NormalTok{,}\DecValTok{3}\NormalTok{,}\DecValTok{2}\NormalTok{,}\DecValTok{1}\NormalTok{,}\DecValTok{2}\NormalTok{,}\DecValTok{2}\NormalTok{,}\DecValTok{8}\NormalTok{,}\DecValTok{5}\NormalTok{,}\DecValTok{2}\NormalTok{,}\DecValTok{1}\NormalTok{,}\DecValTok{1}\NormalTok{,}\DecValTok{6}\NormalTok{,}\DecValTok{4}\NormalTok{,}\DecValTok{3}\NormalTok{)}
\end{Highlighting}
\end{Shaded}

\begin{enumerate}
\def\labelenumi{\alph{enumi}.}
\tightlist
\item
  Construct a Metropolis Hasting sampler that jointly proposes (a,b)
  from a bivariate normal distribution with the current values of both
  parameters as the mean. Begin with a proposal distribution that has a
  diagonal covariance matrix with 0.01 on both diagonal elements.
\end{enumerate}

Use Shaby and Wells' log-adaptive tuning approach to adaptively tune the
proposal distribution, with adaptation happening every 100 MCMC
iterations. Run this until you are sure that your algorithm converges to
the stationary distribution.

\begin{Shaded}
\begin{Highlighting}[]
\CommentTok{\# set for reproducibility}
\FunctionTok{set.seed}\NormalTok{(}\DecValTok{1234}\NormalTok{)}

\CommentTok{\# Libraries}
\FunctionTok{library}\NormalTok{(coda)}
\FunctionTok{library}\NormalTok{(mvtnorm)}
\FunctionTok{library}\NormalTok{(maptools)}
\end{Highlighting}
\end{Shaded}

\begin{verbatim}
## Loading required package: sp
\end{verbatim}

\begin{verbatim}
## Checking rgeos availability: FALSE
## Please note that 'maptools' will be retired during 2023,
## plan transition at your earliest convenience;
## some functionality will be moved to 'sp'.
##      Note: when rgeos is not available, polygon geometry     computations in maptools depend on gpclib,
##      which has a restricted licence. It is disabled by default;
##      to enable gpclib, type gpclibPermit()
\end{verbatim}

\begin{Shaded}
\begin{Highlighting}[]
\FunctionTok{library}\NormalTok{(maps)}
\FunctionTok{library}\NormalTok{(mgcv)}
\end{Highlighting}
\end{Shaded}

\begin{verbatim}
## Loading required package: nlme
\end{verbatim}

\begin{verbatim}
## This is mgcv 1.8-41. For overview type 'help("mgcv-package")'.
\end{verbatim}

\begin{Shaded}
\begin{Highlighting}[]
\FunctionTok{library}\NormalTok{(glmnet)}
\end{Highlighting}
\end{Shaded}

\begin{verbatim}
## Loading required package: Matrix
\end{verbatim}

\begin{verbatim}
## Loaded glmnet 4.1-6
\end{verbatim}

\begin{Shaded}
\begin{Highlighting}[]
\FunctionTok{library}\NormalTok{(MASS)}
\FunctionTok{library}\NormalTok{(nimble)}
\end{Highlighting}
\end{Shaded}

\begin{verbatim}
## nimble version 0.13.1 is loaded.
## For more information on NIMBLE and a User Manual,
## please visit https://R-nimble.org.
## 
## Note for advanced users who have written their own MCMC samplers:
##   As of version 0.13.0, NIMBLE's protocol for handling posterior
##   predictive nodes has changed in a way that could affect user-defined
##   samplers in some situations. Please see Section 15.5.1 of the User Manual.
\end{verbatim}

\begin{verbatim}
## 
## Attaching package: 'nimble'
\end{verbatim}

\begin{verbatim}
## The following object is masked from 'package:stats':
## 
##     simulate
\end{verbatim}

\begin{Shaded}
\begin{Highlighting}[]
\CommentTok{\# \# Source fun}
\CommentTok{\# source("Homework3/mcmc.lm.rr.r")}

\DocumentationTok{\#\# Define the log{-}likelihood function}
\NormalTok{loglike }\OtherTok{\textless{}{-}} \ControlFlowTok{function}\NormalTok{(a, b, x, y) \{}
  \FunctionTok{return}\NormalTok{(}\FunctionTok{sum}\NormalTok{(}\FunctionTok{dpois}\NormalTok{(y, }\FunctionTok{exp}\NormalTok{(a }\SpecialCharTok{+}\NormalTok{ b}\SpecialCharTok{*}\NormalTok{x), }\AttributeTok{log=}\ConstantTok{TRUE}\NormalTok{)))}
\NormalTok{\}}

\CommentTok{\# Set up the prior distribution parameters}
\NormalTok{prior\_mean }\OtherTok{\textless{}{-}} \FunctionTok{c}\NormalTok{(}\DecValTok{0}\NormalTok{, }\DecValTok{0}\NormalTok{)}
\NormalTok{prior\_var }\OtherTok{\textless{}{-}} \FunctionTok{diag}\NormalTok{(}\DecValTok{2}\NormalTok{)}\SpecialCharTok{*}\DecValTok{4}

\CommentTok{\# Set up the proposal distribution}
\NormalTok{proposal\_var }\OtherTok{\textless{}{-}} \FunctionTok{diag}\NormalTok{(}\DecValTok{2}\NormalTok{)}\SpecialCharTok{*}\FloatTok{0.01}

\CommentTok{\# Set up the initial values}
\NormalTok{a }\OtherTok{\textless{}{-}} \DecValTok{0}
\NormalTok{b }\OtherTok{\textless{}{-}} \DecValTok{0}

\CommentTok{\# Set up the MCMC algorithm}
\NormalTok{niter }\OtherTok{\textless{}{-}} \DecValTok{10000}
\NormalTok{samples }\OtherTok{\textless{}{-}} \FunctionTok{matrix}\NormalTok{(}\DecValTok{0}\NormalTok{, niter, }\DecValTok{2}\NormalTok{)}
\NormalTok{accept }\OtherTok{\textless{}{-}} \DecValTok{0}
\NormalTok{cov\_mat }\OtherTok{\textless{}{-}}\NormalTok{ proposal\_var}
\NormalTok{tune\_interval }\OtherTok{\textless{}{-}} \DecValTok{100}
\NormalTok{tune\_count }\OtherTok{\textless{}{-}} \DecValTok{0}
\NormalTok{tune\_iter }\OtherTok{\textless{}{-}}\NormalTok{ tune\_interval}

\CommentTok{\# Run the MCMC algorithm}
\ControlFlowTok{for}\NormalTok{(i }\ControlFlowTok{in} \DecValTok{1}\SpecialCharTok{:}\NormalTok{niter) \{}
  \CommentTok{\# Generate a proposed value for a and b}
\NormalTok{  proposal }\OtherTok{\textless{}{-}} \FunctionTok{mvrnorm}\NormalTok{(}\DecValTok{1}\NormalTok{, }\FunctionTok{c}\NormalTok{(a, b), cov\_mat)}
\NormalTok{  a\_prop }\OtherTok{\textless{}{-}}\NormalTok{ proposal[}\DecValTok{1}\NormalTok{]}
\NormalTok{  b\_prop }\OtherTok{\textless{}{-}}\NormalTok{ proposal[}\DecValTok{2}\NormalTok{]}
  
  \CommentTok{\# Calculate the log{-}likelihoods for the current and proposed values}
\NormalTok{  loglike\_curr }\OtherTok{\textless{}{-}} \FunctionTok{loglike}\NormalTok{(a, b, x, y)}
\NormalTok{  loglike\_prop }\OtherTok{\textless{}{-}} \FunctionTok{loglike}\NormalTok{(a\_prop, b\_prop, x, y)}
  
  \CommentTok{\# Calculate the log{-}prior densities for the current and proposed values}
\NormalTok{  logprior\_curr }\OtherTok{\textless{}{-}} \FunctionTok{dmvnorm}\NormalTok{(}\FunctionTok{c}\NormalTok{(a, b), prior\_mean, prior\_var, }\AttributeTok{log=}\ConstantTok{TRUE}\NormalTok{)}
\NormalTok{  logprior\_prop }\OtherTok{\textless{}{-}} \FunctionTok{dmvnorm}\NormalTok{(}\FunctionTok{c}\NormalTok{(a\_prop, b\_prop), prior\_mean, prior\_var, }\AttributeTok{log=}\ConstantTok{TRUE}\NormalTok{)}
  
  \CommentTok{\# Calculate the log{-}acceptance ratio}
\NormalTok{  log\_accept\_ratio }\OtherTok{\textless{}{-}}\NormalTok{ loglike\_prop }\SpecialCharTok{+}\NormalTok{ logprior\_prop }\SpecialCharTok{{-}}\NormalTok{ loglike\_curr }\SpecialCharTok{{-}}\NormalTok{ logprior\_curr}
  
  \CommentTok{\# Accept or reject the proposed value}
  \ControlFlowTok{if}\NormalTok{ (}\FunctionTok{log}\NormalTok{(}\FunctionTok{runif}\NormalTok{(}\DecValTok{1}\NormalTok{)) }\SpecialCharTok{\textless{}}\NormalTok{ log\_accept\_ratio) \{}
\NormalTok{    a }\OtherTok{\textless{}{-}}\NormalTok{ a\_prop}
\NormalTok{    b }\OtherTok{\textless{}{-}}\NormalTok{ b\_prop}
\NormalTok{    accept }\OtherTok{\textless{}{-}}\NormalTok{ accept }\SpecialCharTok{+} \DecValTok{1}
\NormalTok{  \}}
  
  \CommentTok{\# Save the current values of a and b}
\NormalTok{  samples[i, }\DecValTok{1}\NormalTok{] }\OtherTok{\textless{}{-}}\NormalTok{ a}
\NormalTok{  samples[i, }\DecValTok{2}\NormalTok{] }\OtherTok{\textless{}{-}}\NormalTok{ b}
  
  \CommentTok{\# Update the covariance matrix and tune the proposal distribution}
  \ControlFlowTok{if}\NormalTok{ (i }\SpecialCharTok{==}\NormalTok{ tune\_iter) \{}
    \CommentTok{\# Calculate the acceptance rate}
\NormalTok{    acceptance\_rate }\OtherTok{\textless{}{-}}\NormalTok{ accept}\SpecialCharTok{/}\NormalTok{tune\_interval}
    
    \CommentTok{\# Update the covariance matrix}
\NormalTok{    cov\_mat }\OtherTok{\textless{}{-}} \FunctionTok{cov}\NormalTok{(samples[(i}\SpecialCharTok{{-}}\NormalTok{tune\_interval}\SpecialCharTok{+}\DecValTok{1}\NormalTok{)}\SpecialCharTok{:}\NormalTok{i,])}
    
    \CommentTok{\# Tune the proposal distribution}
    \ControlFlowTok{if}\NormalTok{ (acceptance\_rate }\SpecialCharTok{\textless{}} \FloatTok{0.2}\NormalTok{) \{}
\NormalTok{      proposal\_var }\OtherTok{\textless{}{-}}\NormalTok{ proposal\_var}\SpecialCharTok{/}\DecValTok{2}
\NormalTok{    \} }\ControlFlowTok{else} \ControlFlowTok{if}\NormalTok{ (acceptance\_rate }\SpecialCharTok{\textgreater{}} \FloatTok{0.3}\NormalTok{) \{}
\NormalTok{      proposal\_var }\OtherTok{\textless{}{-}}\NormalTok{ proposal\_var}\SpecialCharTok{*}\DecValTok{2}
\NormalTok{    \}}
    
    \CommentTok{\# Reset the acceptance count}
\NormalTok{    accept }\OtherTok{\textless{}{-}} \DecValTok{0}
    
    \CommentTok{\# Increment the tune count and tune iteration}
\NormalTok{    tune\_count }\OtherTok{\textless{}{-}}\NormalTok{ tune\_count}
\NormalTok{  \}\}  }
    
\CommentTok{\# Print summary of results}
\FunctionTok{cat}\NormalTok{(}\StringTok{"Mean of a:"}\NormalTok{, }\FunctionTok{mean}\NormalTok{(samples[,}\DecValTok{1}\NormalTok{]), }\StringTok{"}\SpecialCharTok{\textbackslash{}n}\StringTok{"}\NormalTok{)}
\end{Highlighting}
\end{Shaded}

\begin{verbatim}
## Mean of a: 2.033761
\end{verbatim}

\begin{Shaded}
\begin{Highlighting}[]
\FunctionTok{cat}\NormalTok{(}\StringTok{"Mean of b:"}\NormalTok{, }\FunctionTok{mean}\NormalTok{(samples[,}\DecValTok{2}\NormalTok{]), }\StringTok{"}\SpecialCharTok{\textbackslash{}n}\StringTok{"}\NormalTok{)}
\end{Highlighting}
\end{Shaded}

\begin{verbatim}
## Mean of b: -0.06873941
\end{verbatim}

\begin{Shaded}
\begin{Highlighting}[]
\FunctionTok{cat}\NormalTok{(}\StringTok{"95\% credible interval for a:"}\NormalTok{, }\FunctionTok{quantile}\NormalTok{(samples[,}\DecValTok{1}\NormalTok{], }\FunctionTok{c}\NormalTok{(}\FloatTok{0.025}\NormalTok{, }\FloatTok{0.975}\NormalTok{)), }\StringTok{"}\SpecialCharTok{\textbackslash{}n}\StringTok{"}\NormalTok{)}
\end{Highlighting}
\end{Shaded}

\begin{verbatim}
## 95% credible interval for a: 1.405703 2.595796
\end{verbatim}

\begin{Shaded}
\begin{Highlighting}[]
\FunctionTok{cat}\NormalTok{(}\StringTok{"95\% credible interval for b:"}\NormalTok{, }\FunctionTok{quantile}\NormalTok{(samples[,}\DecValTok{2}\NormalTok{], }\FunctionTok{c}\NormalTok{(}\FloatTok{0.025}\NormalTok{, }\FloatTok{0.975}\NormalTok{)), }\StringTok{"}\SpecialCharTok{\textbackslash{}n}\StringTok{"}\NormalTok{)}
\end{Highlighting}
\end{Shaded}

\begin{verbatim}
## 95% credible interval for b: -0.1085646 -0.02754659
\end{verbatim}

\begin{Shaded}
\begin{Highlighting}[]
\DocumentationTok{\#\#\# Accept/Reject {-}{-}{-}{-}}
\NormalTok{accept}
\end{Highlighting}
\end{Shaded}

\begin{verbatim}
## [1] 4414
\end{verbatim}

\begin{Shaded}
\begin{Highlighting}[]
\CommentTok{\# [1] 4770}

\CommentTok{\# Plot results}
\FunctionTok{par}\NormalTok{(}\AttributeTok{mfrow=}\FunctionTok{c}\NormalTok{(}\DecValTok{2}\NormalTok{,}\DecValTok{1}\NormalTok{))}
\FunctionTok{plot}\NormalTok{(samples[,}\DecValTok{1}\NormalTok{], }\AttributeTok{type=}\StringTok{"l"}\NormalTok{, }\AttributeTok{ylab=}\StringTok{"a"}\NormalTok{, }\AttributeTok{main=}\StringTok{"Traceplot of a"}\NormalTok{)}
\FunctionTok{plot}\NormalTok{(samples[,}\DecValTok{2}\NormalTok{], }\AttributeTok{type=}\StringTok{"l"}\NormalTok{, }\AttributeTok{ylab=}\StringTok{"b"}\NormalTok{, }\AttributeTok{main=}\StringTok{"Traceplot of b"}\NormalTok{)}
\end{Highlighting}
\end{Shaded}

\includegraphics{Homework3_VAW_files/figure-latex/unnamed-chunk-2-1.pdf}

\hypertarget{question-2}{%
\section{Question 2}\label{question-2}}

Specify a reasonable prior distribution for the following situations.

\hypertarget{question-2a.}{%
\section{Question 2a.}\label{question-2a.}}

\begin{enumerate}
\def\labelenumi{\alph{enumi}.}
\tightlist
\item
  Your data are Bernoulli distributed, with shared probability of
  success p.~Your goal for a prior distribution on p is that your prior
  is vague, giving equal probability for any valid value for p.~
\end{enumerate}

\hypertarget{answer}{%
\section{answer}\label{answer}}

p \textasciitilde{} Beta(1,1) where p is the probability of success in a
Bernoulli trial, and the parameters of the Beta distribution are both
equal to 1. This prior has a simple closed-form expression for the
posterior distribution, which is also a Beta distribution with updated
parameters that depend on the number of successes and failures in the
data.

\hypertarget{question-2b.}{%
\section{Question 2b.}\label{question-2b.}}

Your data are Bernoulli distributed, with shared probability of success
p.~Your goal for a prior distribution on p is to specify a prior
distribution that has a 90\% probability that p is between 0 and 0.5,
with a 10\% probability that p is greater than 0.5. \# answer Beta(a,b)
where a and b are chosen such that 90\% of the prior probability mass
falls in the interval {[}0,0.5{]} and 10\% of the mass falls in the
interval (0.5,1{]}.

One way to achieve this is to set a = 0.9 and b = 1.8, which corresponds
to a prior mean of 0.33 and a prior variance of 0.023. This choice of
parameters gives a prior probability of 0.9 that p is between 0 and 0.5,
and a prior probability of 0.1 that p is greater than 0.5.

The Beta(a,b) prior distribution has a probability density function that
is proportional to p\^{}(a-1) * (1-p)\^{}(b-1).

\hypertarget{question-2c.}{%
\section{Question 2c.}\label{question-2c.}}

Your data are Bernoulli distributed, with shared probability of success
p.~Your goal for a prior distribution on p is to specify a prior
distribution that has a 50\% probability that p is exactly 0, and a 50\%
probability that p is somewhere between 0 and 1, with equal probability
given. Normal? \# answer a mixture distribution that assigns 50\%
probability to a point mass at 0 and 50\% probability to a continuous
uniform distribution on the interval {[}0,1{]}. f(p) = 0.5 * delta(0) +
0.5 * U(0,1) where delta(0) represents a point mass at 0 and U(0,1)
represents a uniform distribution on the interval {[}0,1{]}.

Alternatively, you can use a truncated normal distribution as a prior
for p here. You can set the mean of the normal distribution to be less
than zero to ensure that the prior assigns a 50\% probability to p being
exactly 0. The variance of the normal distribution can be chosen to
control the spread of the prior. Finally, you can truncate the normal
distribution at 0 and 1 to ensure that the prior assigns equal
probability to p being between 0 and 1.

\hypertarget{question-2d.}{%
\section{Question 2d.}\label{question-2d.}}

Your data are classic linear regression data, with response and
predictor variables. The regression parameter \beta for one parameter of
interest has been well studied in the literature. Specify a prior for
this distribution which allows for any real number, but which has a 95\%
prior probability of being between -0.2 and -0.1. \# answer empirical
bayes w. normal prior A normal distribution with mean -0.15 and a chosen
sd from the literature would allow for any real number Ex: normal
distribution with mean -0.15 and standard deviation 0.05, which would
give a 95\% prior probability of being between -0.2 and -0.1.

\hypertarget{question-2e.}{%
\section{Question 2e.}\label{question-2e.}}

\begin{verbatim}
Your data come from a physical process where you know that your parameter must be between 0 and 2.  A previous study estimated the parameter as being very close to 2.  Specify a prior that respects the required physical constraints, and also places a 75% probability that the parameter is between 1.8 and 2. 
\end{verbatim}

\hypertarget{answer-1}{%
\section{answer}\label{answer-1}}

\begin{verbatim}
a beta distribution with parameters alpha = 4 and beta = 1.25 would have a mean of 0.76 and a mode of 0.94, with 75% of the distribution falling between 1.8 and 2.
\end{verbatim}

\end{document}
